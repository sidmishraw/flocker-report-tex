%
% flocker.tex
% @author Sidharth Mishra
% @description 
% @copyright  BSD 3-Clause License
%
% Copyright (c) 2018, Sidharth Mishra
% All rights reserved.
%
% Redistribution and use in source and binary forms, with or without
% modification, are permitted provided that the following conditions are met:
%
% * Redistributions of source code must retain the above copyright notice, this
%  list of conditions and the following disclaimer.
%
% * Redistributions in binary form must reproduce the above copyright notice,
%  this list of conditions and the following disclaimer in the documentation
%  and/or other materials provided with the distribution.
%
% * Neither the name of the copyright holder nor the names of its
%  contributors may be used to endorse or promote products derived from
%  this software without specific prior written permission.
%
% THIS SOFTWARE IS PROVIDED BY THE COPYRIGHT HOLDERS AND CONTRIBUTORS "AS IS"
% AND ANY EXPRESS OR IMPLIED WARRANTIES, INCLUDING, BUT NOT LIMITED TO, THE
% IMPLIED WARRANTIES OF MERCHANTABILITY AND FITNESS FOR A PARTICULAR PURPOSE ARE
% DISCLAIMED. IN NO EVENT SHALL THE COPYRIGHT HOLDER OR CONTRIBUTORS BE LIABLE
% FOR ANY DIRECT, INDIRECT, INCIDENTAL, SPECIAL, EXEMPLARY, OR CONSEQUENTIAL
% DAMAGES (INCLUDING, BUT NOT LIMITED TO, PROCUREMENT OF SUBSTITUTE GOODS OR
% SERVICES; LOSS OF USE, DATA, OR PROFITS; OR BUSINESS INTERRUPTION) HOWEVER
% CAUSED AND ON ANY THEORY OF LIABILITY, WHETHER IN CONTRACT, STRICT LIABILITY,
% OR TORT (INCLUDING NEGLIGENCE OR OTHERWISE) ARISING IN ANY WAY OUT OF THE USE
% OF THIS SOFTWARE, EVEN IF ADVISED OF THE POSSIBILITY OF SUCH DAMAGE.
%
% @created Sat Mar 31 2018 22:05:54 GMT-0700 (PDT)
% @last-modified Sat Mar 31 2018 23:45:07 GMT-0700 (PDT)
%


\documentclass[../main]{subfiles}

\begin{document}

\section{Flocker}

 {\em Flocker} is the core component responsible for the visualization part of the {\em flocking} simulation. 
 It has been implemented using the {\em p5.js}\cite{p5js} {\em Javascript} library. 
 
 The source-code for {\em Flocker} has been written in {\em Typescript}(TS) which has been compiled and bundled into a single \code{.js} file using the {\em Webpack} tool.

 The central idea for implementing a simulation involving agents is coming up with a object for each agent. In our case, the agents are {\em swallows}. Each swallow is given an {\em image}, {\em position}, {\em velocity}, {\em acceleration}. When a {\em swallow} spawns -- gets added to the simulation -- it is assigned a random position. Similarly, each {\em swallow} is assigned a random {\em velocity} in a random direction. This is achieved by using {\em p5.js} \code{Vector} objects. However, initially, each {\em swallow} has \code{0} {\em acceleration} -- initially, no forces are acting on the {\em swallow}.
 
\end{document}